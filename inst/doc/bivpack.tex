\documentclass[12pt,oneside]{article}
\usepackage{pdfsync}
\usepackage{pdfpages}
\usepackage[top=1.25in,bottom=1in,left=1.5in,right=1in]{geometry}
\linespread{1.5}
\usepackage{graphicx}

\pagenumbering{arabic}
%\setcounter{page}{0}

\usepackage{natbib}
\usepackage[breaklinks,bookmarks=false]{hyperref}
\usepackage{verbatim}
\usepackage{enumerate}
\usepackage{statex2}
%\usepackage{shortvrb}
%\MakeShortVerb{!}

%\usepackage{hyperref}

\pagestyle{headings}

\renewcommand*{\approx}{\;\stackrel{\mb{\text{.}}}{~}\;}
\newcommand*{\approxequal}{\;\stackrel{~}{\mb{\text{=}}}\;}
\newcommand*{\data}{\mb{\text{data}}}
%\renewcommand*{\and}{\mb{\text{and}}}
\newcommand*{\ad}{\stackrel{\bm{.}}{~}}

\title{BIVpack: Bayesian methods for IV regression}
\author{Purushottam Laud, Rodney Sparapani,\\
Jessica Pruszynski and Robert McCulloch}
\date{\today}

\definecolor{color0}{rgb}{0,0,0}
\definecolor{color1}{rgb}{1,0,0}
\definecolor{color2}{rgb}{0,1,0}
\definecolor{color3}{rgb}{0,0,1}

\begin{document}
\maketitle

\section{BIVpack: Bayesian methods for IV regression}

BIVpack is an R package implementing Bayesian methods for IV
regression.  BIVpack supports both parametric and nonparametric
calculations; the parametric calculations are purely of academic
interest only.  Paraphrasing \citet{ImbeAngr94}: Causal inference of
an observational study requires the nonparametric identification of
treatment effects without relying on functional form restrictions or
distributional assumptions.

On the MCW Division of Biostatistics web page, there is a link to
BIVpack at:
[\url{http://www.mcw.edu/biostatistics/statisticalresources/CollaborativeSoftware.htm}].  This link will take you to R-Forge: an open platform for the
development of R packages, R-related software and further projects. It
is based on SVN \citep{CollFitz11} and web technology to provide R
packages, mailing lists, bug tracking, message boards/forums, site
hosting, permanent file archival, full backups, and total web-based
administration.  The permanent BIVpack R-forge location is
[\url{http://r-forge.r-project.org/projects/bivpack}].

\section{Background}

Statisticians have a long history of using specialized, interactive
programming environments for data processing and statistical analysis
(since modern general purpose interactive languages like Perl and Python 
do not readily provide the mathematical and statistical building
blocks statisticians require).
According to the TIOBE popularity rankings of programming languages
[\url{http://www.tiobe.com/index.php/content/paperinfo/tpci}], 
R \citep{R} is the second most popular statistical programming language.

R is an interpreted, object-oriented language and environment for
statistical computing and graphics; it is a free software project
falling under the GNU Public License (GPL).  R is based on other GPL
technologies like the GNU Compiler Collection (GCC) of C, C++ and
Fortran compilers: [\url{http://gcc.gnu.org}].  R provides the basis
upon which over 5000 R packages have been created to perform ever more
specialized purposes: [\url{http://lib.stat.cmu.edu/R/CRAN}].

BIVpack was created in this nutrient rich gene pool.  BIVpack relies
heavily on two R packages: Rcpp,
[\url{http://lib.stat.cmu.edu/R/CRAN/web/packages/Rcpp}],
and RcppEigen,
[\url{http://lib.stat.cmu.edu/R/CRAN/web/packages/RcppEigen}].
Rcpp provides an interface between the relatively slow, interactive
performance of object-oriented R and fast, efficient, object-oriented,
C++ compiled code.  RcppEigen uses Rcpp to integrate R with
Eigen, [\url{http://eigen.tuxfamily.org}]: a C++ template library for
linear algebra, e.g.\ matrices, vectors, numerical solvers, and related
algorithms.

\section{Parametric Models}

BIVpack provides 3 functions for estimating parametric models:
\texttt{nniv} for a numeric treatment and outcome;
\texttt{bniv} for a binary treatment and a numeric outcome; and
\texttt{bbiv} for a binary treatment and outcome.  Each of these
functions takes 3 arguments; \texttt{info}, \texttt{data} and
\texttt{mcmc}; and returns the posterior samples of the parameters
as a matrix.

\texttt{info} is a list of parameters
with their initial values and prior parameter settings.
\texttt{data} is a list containing an \texttt{X} matrix for the
confounders (if no confounders are present, then provide a matrix
with a column of zeros), a \texttt{Z} matrix for the instruments, a
treatment vector \texttt{t} and an outcome vector \texttt{y}:
for a binary treatment the vector should be called \texttt{tbin}
and for a binary outcome \texttt{ybin}.  \texttt{mcmc} is a list
of the Markov chain Monte Carlo parameters:  \texttt{M} for the
length of the chain, \texttt{burnin} for the amount to discard
from the beginning and \texttt{thin} for reducing auto-correlation
by only keeping a fraction of the chain.

In BIVpack, there is an example provided in \texttt{man/BIVpackage.Rd}\ ;
we will demonstrate these functions via excerpts from this file.

\subsection{\texttt{nniv}}

\begin{verbatim}
require(BIVpack)

N <- 10
p <- 0
q <- 1
p1 <- max(1, p)
r <- p1+q
s <- p1+r

gamma <- 0
delta <- 4 
eta  <- 0
beta <- 0.5
mu   <- 0
rho  <- 0.6

mcmc <- list(M=1, burnin=0, thin=1)

info <- list(theta=list(init=c(rep(gamma, p1), rep(delta, q), rep(eta, p1)),
               prior=list(mean=rep(0., s),
                 prec=diag(0.001, s))),
             beta=list(init=beta, prior=list(mean=0., prec=0.001)),
             Tprec=list(init=solve(matrix(c(1, rho, rho, 1), 2, 2)),
               prior=list(nu=4, Psi=diag(1, 2, 2))),
             mu=list(init=c(mu, mu),
               prior=list(mean=c(0., 0.), prec=diag(0.001, 2))))

data <-  list(X=matrix(0, nrow=N, ncol=p1),
             Z=matrix(c(-0.24146164, -0.29673723, -0.27538621,
               0.41463628, 0.39023100, -0.22045922, -0.07062149,
               -0.22595298, 0.01247487, -0.14472589), nrow=N, ncol=q),
             t=c(-2.01322819, -2.04167660, -0.56128516,
               0.20783192, 0.31477076, -1.41477107, -0.38701899,
               -0.59955150, 0.01197733, -0.79804809),
             y=c(-1.9924944, -1.9345279, -1.3781082, -0.7646928,
               -0.2881649, 0.1545577, -0.6114224, -0.3703420,
               0.2320320, 0.7451867))

set.seed(42)
nniv(info, data, mcmc)
##should produce approx...
##        gamma1   delta1     eta1      beta        mu1         mu2      T11       T12       T21      T22
## [1,] -17.85732 4.008851 20.01287 0.8217393 -0.3821324 -0.05601638 3.108135 -1.306804 -1.306804 2.188039
\end{verbatim}
\pagebreak

\subsection{\texttt{bniv}}

\begin{verbatim}
info <- list(beta=list(init=0.,
               prior=list(mean=0., prec=0.001)),
             rho=list(init=0.6),
             mu=list(init=c(0.,0.),
               prior=list(mean=c(0.,0.), prec=diag(0.001,2))),
             tau=list(init=1., prior=list(alpha0=0.1, lambda0=0.1)),
             theta=list(init=c(rep(0., p1), rep(0., q), rep(0., p1)),
               prior=list(mean=rep(0.,s), prec=diag(0.001, s))))

data <-  list(X=matrix(0, nrow=N, ncol=p1),
             Z=matrix(c(-0.24146164, -0.29673723, -0.27538621,
               0.41463628, 0.39023100, -0.22045922, -0.07062149,
               -0.22595298, 0.01247487, -0.14472589), nrow=N, ncol=q),
             tbin=as.integer(c(0, 0, 0, 1, 1, 0, 0, 0, 1, 0)),
             y=c(-1.9924944, -1.9345279, -1.3781082, -0.7646928,
               -0.2881649, 0.1545577, -0.6114224, -0.3703420,
               0.2320320, 0.7451867))

set.seed(42)
(par.post <- bniv(info, data, mcmc))
##should produce approx...
##       gamma1   delta1     eta1       beta       mu1        mu2       rho     s2sq
## [1,] 47.79852 1.231516 63.82816 -0.1581546 0.1076077 -0.2405642 0.4071552 0.626958
\end{verbatim}
\pagebreak

\subsection{\texttt{bbiv}}

\begin{verbatim}
info <- list(theta=list(init=c(rep(gamma, p1), rep(delta, q), rep(eta, p1)),
               prior=list(mean=rep(0., s),
                 prec=diag(0.001, s))),
             beta=list(init=beta, prior=list(mean=0., prec=0.001)),
             rho=list(init=rho),
             mu=list(init=c(mu, mu),
               prior=list(mean=c(0., 0.), prec=diag(0.001, 2))))

data <-  list(X=matrix(0, nrow=N, ncol=p1),
             Z=matrix(c(-0.24146164, -0.29673723, -0.27538621,
               0.41463628, 0.39023100, -0.22045922, -0.07062149,
               -0.22595298, 0.01247487, -0.14472589), nrow=N, ncol=q),
             tbin=as.integer(c(0, 0, 0, 1, 1, 0, 0, 0, 1, 0)),
             ybin=as.integer(c(0, 0, 0, 0, 0, 1, 0, 0, 1, 1)))

set.seed(42)
(par.post <- bbiv(info, data, mcmc))
##should produce approx...
##      gamma1   delta1      eta1      beta        mu1        mu2       rho
## [1,] 3.41756 3.697175 -15.94475 0.1152464 -0.7233647 -0.8709668 0.3693283
\end{verbatim}

2SLS estimates the IVE according to \citet{ImbeAngr94}.  We provide a function
to compute the IVE from the binary treatment and outcome model.
\begin{verbatim}
bbivE(par.post[1, ], data$Z, data$X)
##should produce approx...
## 0.0995336
\end{verbatim}

\section{Nonparametric Models}

Technically, the models we present are semiparametrics models, but we stick with
the nonparametric nomenclature in this document for convenience.
We do not provide a function for nonparametric model with 
a numeric treatment and outcome; for that see the bayesm package 
[\url{http://lib.stat.cmu.edu/R/CRAN/web/packages/bayesm}].

BIVpack provides 2 functions for estimating nonparametric models:
\texttt{bnivDPM} for a binary treatment and a numeric outcome; and
\texttt{bbivDPM} for a binary treatment and outcome.  Each of these
functions takes 3 arguments; \texttt{info}, \texttt{data} and
\texttt{mcmc}; and returns the posterior samples of the parameters
as a list.  Besides being nonparametric, these functions provide
smarter handling of the \texttt{info} and \texttt{data} parameters.

We follow the advice of \citet{GelmJaku08}.  There
are two parts relevant to our models:  weakly informative prior
parameters and data standardization.  If you pass a \texttt{NULL} 
list for the \texttt{info} parameter, then a default prior parameterization is 
constructed for you.  And, if you pass the optional parameter
\texttt{stdize=TRUE}, then data standardization is performed
and a back-transformation is employed.


\subsection{\texttt{bnivDPM}}

\begin{verbatim}
data <-  list(X=matrix(0, nrow=N, ncol=p1),
             Z=matrix(c(-0.24146164, -0.29673723, -0.27538621,
               0.41463628, 0.39023100, -0.22045922, -0.07062149,
               -0.22595298, 0.01247487, -0.14472589), nrow=N, ncol=q),
             tbin=as.integer(c(0, 0, 0, 1, 1, 0, 0, 0, 1, 0)),
             y=c(-1.9924944, -1.9345279, -1.3781082, -0.7646928,
               -0.2881649, 0.1545577, -0.6114224, -0.3703420,
               0.2320320, 0.7451867))

info <- list(beta=list(init=0, prior=list(mean=0, prec=0.001)),
             rho=list(init=0),
             mu=list(init=c(0, 0)),
             tau=list(init=1),
             theta=list(init=rep(0, s),
               prior=list(mean=rep(0, s),
                   prec=diag(c(rep(0.04, r), rep(0.001, p1)), s, s))),
             dpm=list(m=as.integer(3),
                 alpha=list(fixed=as.integer(0), init=1, prior=list(a=3, b=4)),
                 C=as.integer(0*(1:N)), states=as.integer(N),
                 prior=list(mu0=c(0, 0), T0=diag(0.001, 2), S0=diag(1000, 2),
                    alpha0=0.5, lambda0=0.4)))

set.seed(42)
(non.post <- bnivDPM(info, data, mcmc))
##should produce approx...
## [[1]]
## [[1]]$beta
## [1] 0.1070952

## [[1]]$theta
## [1]  7.557610  2.473789 63.828162

## [[1]]$C
##  [1] 0 0 0 0 0 0 0 0 0 0

## [[1]]$phi
##             [,1]      [,2]       [,3]    [,4]
## [1,] -0.07192311 -1.002996 -0.3468734 1.80566

## [[1]]$states
## [1] 10

## [[1]]$alpha
## [1] 0.1844611

set.seed(42)
(non.post <- bnivDPM(NULL, data, mcmc))
##should produce approx...
## same as above
\end{verbatim}

\subsection{\texttt{bbivDPM}}

\begin{verbatim}
data <-  list(X=matrix(0, nrow=N, ncol=p1),
             Z=matrix(c(-0.24146164, -0.29673723, -0.27538621,
               0.41463628, 0.39023100, -0.22045922, -0.07062149,
               -0.22595298, 0.01247487, -0.14472589), nrow=N, ncol=q),
             tbin=as.integer(c(0, 0, 0, 1, 1, 0, 0, 0, 1, 0)),
             ybin=as.integer(c(0, 0, 0, 0, 0, 1, 0, 0, 1, 1)))

info <- list(beta=list(init=0, prior=list(mean=0, prec=1)),
             rho=list(init=0),
             mu=list(init=c(0, 0)),
             tau=list(init=1),
             theta=list(init=rep(0, s),
               prior=list(mean=rep(0, s), prec=diag(0.04, s))),
             dpm=list(m=as.integer(3),
                 alpha=list(fixed=as.integer(0), init=1, prior=list(a=3, b=4)),
                 C=as.integer(0*(1:N)), states=as.integer(N),
               prior=list(mu0=c(0, 0), T0=diag(1, 2), S0=diag(1, 2),
                    alpha0=0.5, lambda0=0.4)))

set.seed(42)
(non.post <- bbivDPM(info, data, mcmc))
##should produce approx...
## [[1]]
## [[1]]$beta
## [1] -0.2440909

## [[1]]$theta
## [1]  0.5403636  1.4322290 -2.5210857

## [[1]]$C
##  [1] 0 0 0 0 1 0 1 0 0 0

## [[1]]$phi
##           [,1]       [,2]  [,3]
## [1,] 0.5215824 -0.9161644 -0.39
## [2,] 0.1711343 -0.9186419 -0.43

## [[1]]$states
## [1] 8 2

## [[1]]$alpha
## [1] 1.029461

set.seed(42)
(non.post <- bbivDPM(NULL, data, mcmc))
##should produce approx...
## same as above

set.seed(42)
bbivDPM(info, data, mcmc, stdize=TRUE)
##should produce approx...
## [[1]]
## [[1]]$beta
## [1] -0.2440909

## [[1]]$theta
## [1]  0.2701818  1.3766962 -1.2605428

## [[1]]$C
##  [1] 0 0 0 0 1 0 1 0 0 0

## [[1]]$phi
##           [,1]       [,2]  [,3]
## [1,] 0.5256154 -0.9161644 -0.39
## [2,] 0.2071443 -0.9162833 -0.42

## [[1]]$states
## [1] 8 2

## [[1]]$alpha
## [1] 1.029461

bbivE(non.post[[1]], data$Z, data$X)
##should produce approx...
##  -0.1687742
\end{verbatim}

\bibliographystyle{chicago}
\bibliography{bivpack.bib}

\end{document}

